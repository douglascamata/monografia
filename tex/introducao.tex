\chapter{Introdução}

Segundo \citeonline{book-cloud-developing-countries}, muitas instituições de ensino superior e pesquisa
de países em desenvolvimento sofrem com diversos problemas relacionados com o gerenciamento dos
seus recursos computacionais. Eles trabalham com grupos que apresentam 3 características:

\begin{itemize}
    \item
        Não possuem acesso adequado aos recursos computacionais para pesquisa e ensino;
    \item
        Se envolvem em alta carga de atividades de ensino;
    \item
        São formados por poucos membros e possuem um orçamento reduzido.
\end{itemize}

Estas atividades (pesquisa e ensino), ainda apresentam outras importantes características.
Por exemplo, elas fazem um grande uso de recursos computacionais em curtos períodos
de tempo, devido à estrutura dos semestres letivos, enquanto outros são usados
mais regularmente \cite{book-cloud-developing-countries}.

Apesar dos avanços em tecnologia, o compartilhamento e gerenciamentos destes recursos
entre instituições de ensino tem recebido uma grande prioridade, principalmente nas
regiões em desenvolvimento \cite{article-sharing-importance}.
\citeonline{article-cloud-developing-countries}
diz ainda que devido às altas demandas de recursos por estas instituições, os grupos dificilmente
conseguem adquirir ou até mesmo manter recursos computacionais de alta performance.

Muitos estudos mostram que \emph{cloud computing} tem benefícios distintos, como
redução de custos, utilização eficiente de recursos computacionais,
flexibilidade e provisionamento elástico \cite{article-cloud-cost-reduction}. Além disso,
\citeonline{book-cloud-developing-countries} diz que todos estes benefícios se aplicam
às duas principais atividades em instituições de ensino superior: pesquisa e ensino. Ele
também diz que existem muitos outros artigos analisando o quão atrativo é o \emph{cloud computing}
para países em desenvolvimento. Ele cita o exemplo de \citeonline{article-cloud-benefits},
que apresenta vários benefícios, como fácil acesso à infraestrutura com baixo custo, melhorando
os esforços para colaboração e o acesso a hardware e softwares mais recentes. Estes benefícios
foram de fato confirmados pelo trabalho de \citeonline{book-cloud-developing-countries}.

Até mesmo instituições governamentais, como o Governo Federal dos Estados Unidos da América,
identificaram \emph{cloud computing} como uma ótima solução para os problemas
que estavam enfrentando que sua estrutura de tecnologia de informação \cite{federal-cloud-computing}.
Estima-se que cerca de 20, dos seus 80 bilhões de dólares de investimento em TI,
serão utilizados para a migração de seus sistemas para este novo paradigma.

\section{Problemática e justificativa}

Segundo \citeonline{article-cloud-developing-countries}, a característica principal
relacionada a pesquisa em instituições de ensino é que os grupos de pesquisa são
pequenos e não são bem coordenados. Como resultado, os investimentos nas pesquisas
acabam sendo \textit{ad-hoc}, sob demanda e muito frequentemente não coordenados. Ainda
dentro deste assunto, o autor diz que estes grupos tendem a resolver problemas pequenos e
que não são recentes, porque não possuem recursos suficientes para resolver problemas
maiores e mais complexos. Alguns grupos precisam de acesso a computação  de alta
\emph{performance} para simulações e análises de seus modelos. Porém, o orçamento reduzido
acaba limitando muito a performance dos recursos computacionais que a instituição
é capaz de adquirir. Todas estas características culminam nos seguintes problemas:

\begin{itemize}
    \item
        Recursos fragmentados e inadequados para resolução de grandes problemas. Eles
        são divididos em pequenos e desconectados grupos, capazes de resolver somente
        pequenos problemas.
    \item
        Os pesquisadores querem ter total controle sobre seus recursos, principalmente
        para customizá-los para seus próprios problemas. Esta atitude não facilita, mas sim
        atrapalha, o compartilhamento destes recursos.
    \item
        O baixo orçamento torna muito difícil a aquisição de recursos computacionais.
        Porém, a fragmentação e falta de esforços coordenados em dividir e compartilhar
        os recursos já obtidos resultam em uma subutilização destes.
\end{itemize}

Um estudo feito por \citeonline{article-learning-as-a-service} mostra que existem muitos modelos de \emph{cloud computing}
que alcançaram o sucesso dentro de instituições de ensino, utilizando soluções comerciais e não-comerciais, terceirizadas
e não-terceirizadas.
Porém, existem várias ``formas'' de \emph{cloud computing} e é essencial que as instituições de ensino
que desejam utilizar este modelo escolham a forma adequada para melhor satisfazer suas necessidades computacionais.

Pode-se identificar estes mesmos problemas anteriormente mecionados
na Universidade Estadual do Norte Fluminense (UENF). Como exemplo,
há o caso de alguns professores do Centro de Ciências Biológicas, onde eles
possuem seu conjunto de recursos computacionais de alta performance. Estes recursos são
relativamente pequenos para soluções de grandes problemas, com necessidades
de software bem específicas, os professores desejam 100\% destes recursos
disponíveis quando eles desejam executar seus procedimentos e não admitem
que existam outros procedimentos concorrentes além dos seus próprios. Além
destes poblemas, ainda há a dificuldade burocrática e financeira envolvida na
compra de recursos adicionais, pois a compra demora uma quantiadade considerável
de tempo até ser efetivada e o orçamento da universidade dificilmente permite a
compra de equipamentos de última geração e com configurações de topo de linha.

Há também o problema de gerenciamento das aplicações desenvolvidas por alunos e
professores que são disponibilizadas para o público. Atualmente, por exemplo,
os alunos do curso de Ciência da Computação dispõem de apenas um computador
pessoal para ser utilizado como servidor. Diversos problemas são enfrentados
pelos alunos, dentre eles, os mais graves são a correta configuração e
instalação dos programas necessárias para que sua aplicação funcione e ainda
sem interferir com outras aplicações instaladas no mesmo servidor, e questões
relacionadas a segurança: quem deve ter a senha do servidor compartilhado, todos ou só um?
Estes problemas acabam causando uma ilha de conhecimento, pois por diversos motivos,
nem todos oa alunos podem ou precisam aprender isso e apenas um deles é escolhido para fazer
todas as instalações, culminando numa dependência muito grande de uma única pessoa.

\section{Objetivos}

Este trabalho tem como objetivo apresentar um modelo de \emph{cloud computing},
definido pela escolha de uma ferramenta e modelo adequados para sua implantação,
capaz de minimizar estes problemas, e todos os outros já mencionados no início
deste capítulo, dentro da
Universidade Estadual do Norte Fluminense, especialmente para a disponibilização
de projetos desenvolvidos pelos alunos e pesquisadores do Laboratório Ada
Lovelace/LCMAT, mas não limitando-o a este propósito. Tal modelo é concebido
através do \emph{framework} proposto por \citeonline {article-learning-as-a-service}.
Também é um objetivo deste trabalho fazer uma discussão mais profunda sobre os
detalhes técnicos da instalação do modelo, prover instruções para este processo,
analisar as características do produto resultante e do processo, como um todo,
verificar o nível do desenvolvimento do conjunto de ferramentas escolhido.

\section{Metodologia}

Vários trabalhos já aqui mencionados enumeram casos de sucesso, onde o \emph{cloud computing} soluciona os problemas
encontrados por instituições de ensino localizadas em países em desenvolvimento com relação à gerência e aproveitamento
dos seus recursos computacionais. Apesar de todos eles mencionarem que a solução foi alcançada, nenhum deles apresenta
mais a fundo quais foram as ferramentas e as configurações destas que foram utilizadas pelas instituições, muito menos
a razão da escolha de determinada ferramenta em relação a outras e os procedimentos seguidos para a implantação do ambiente.

Este trabalho pretende abordar estes pontos que ainda não foram foco de muita literatura, escolhendo o
\emph{OpenStack}, que é um dos conjuntos de ferramentas para \emph{cloud computing} mais utilizado e
que evolui com mais velocidade, fazendo uma análise dos seus componentes, prosseguindo com sua
instalação em um pequeno ambiente de testes e análise de suas funcionalidades, segurança e facilidades.

\section{Organização}
\label{sec:organizacao}

Este trabalhado será apresentado seguindo a estrutura de capítulos abaixo:

\begin{enumerate}

    \item
        Introdução: onde são brevemente apresentados alguns casos de uso, os objetivos
        e justificativa, e a metodologia de desenvolvimento do mesmo.

    \item
        Fundamentação teórica, onde serão apresentados conceitos, métodos e definições
        relacionadas a \emph{cloud computing}, suas vantagens e desvantagens, assim como
        o estado de arte do tema.

    \item
        Apresentação do experimento, das ferramentas escolhidas para sua execução e das razões que levaram à
        escolha destas em relação a outras por meio de comparações de prós e contras. Enumeração das possíveis
        configurações para instalação, assim como das características ótimas para obter o maior desempenho através
        deste novo paradigma. Inclui também informações instrutivas sobre a configuração destas ferramentas em um ambiente de testes pré-determinado.

    \item
        Descrição do funcionamento das ferramentas no ambiente de teste mencionado no item anterior.
        Verificação das vantagens reais das ferramentas através deste ambiente e comparação com o paradigma de computação tradicional.

    \item
        Conclusões sobre o trabalho, considerações finais e comentários sobre possíveis trabalhos futuros.

\end{enumerate}
