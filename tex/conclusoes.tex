\chapter{Conclusões}
\label{cha:conclusoes}

Pode-se concluir através das análises da bibliografia, das tecnologias e das
ferramentas apresentadas que \emph{cloud computing} é extremamente eficaz em
um ambiente acadêmico. Professores podem ter um rápido acesso a infraestrutura
para poder executar e/ou disponibilizar seus projetos, e alunos para auxílio
na instalação e configuração de ambientes para o aprendizado das mais diversas
tecnologias disponíveis, sem toda a burocracia já conhecida das instituições
públicas de ensino superior
(um servidor pode ser entregue em questão de minutos, em vez de dias, ou até mais),
Isto soluciona o problema alunos do curso de Ciência da Computação, que poderão
instalar seus aplicativos em máquinas virtuais isoladas, sem interferir uns nos
outros e ter que lidar com sistema operacional ``congestionado'' por ter diversos
\emph{softwares} diferentes instalados.
Além disso, toda a arquitetura obtida desta maneira possui um grande potencial
elástico e pode ser manipulada por uma interface web simples, intuitiva, poderosa
e eficaz para que os recursos sejam redimensionados conforme a necessidade do usuário.
É oferecida uma economia para a instituição de ensino, que através da
distribuição dos servidores virtuais nos físicos e seu uso sob-demanda, tem um
melhor proveito dos mesmos. Todas estas conclusões confirmam que os problemas
encontradas pela UENF com a gerencia de seus recursos computacionais podem
ser solucionados inteligentemente e oferecer vantagens para diversos alunos
e professores da universidade.

É visto que a instalação e configuração do próprio \emph{OpenStack} e das
ferramentas que é compõe tem sua complexidade concentrada no modelo de rede:
existem diversas ferramentas de automação que realizam todo processo de instalação
e configuração de pacotes Linux nos servidores de maneira automática e bem
transparente. A Rackspace oferece um ótimo suporte gratuito para auxiliar em
algumas tarefas e tem um completo time experiente em grandes ambientes de
\emph{cloud computing} para consultorias e outros serviços pagos relacionados
a escalabilidade e manutenção dos recursos computacionais necessários, o que é
de suma importância.

Foi constado que a ferramenta está evoluindo em velocidade crescente, com cada
vez mais recursos sendo integrados e desenvolvidos pelos contribuidores do projeto.
Somente durante o desenvolvimento deste trabalho, duas novas versões foram lançadas,
ambas adicionando novas funcionalidades para suprir as necessidades dos seus
usuários. Isto tudo é permitido graças ao desenvolvimento no modelo de software
livre, onde todos podem contribuir com o projeto.

\section{Trabalhos futuros}
\label{sec:trabalhos_futuros}

Este trabalho deixa inúmeros assuntos interessantes para trabalhos futuros, como
uma análise de ferramentas capazes de levar a \emph{cloud} para o nível de
plataforma como serviço (usando ferramentas como as já mencionadas Juju e Tsuru).
Tal serviço é de extrema utilidade para desenvolvedores de aplicações que não
tem conhecimento de gerenciamento de servidores e torna esta tarefa extremamente
simples, rápida e prazerosa. Também seria de grande contribuição uma análise
de performance de diferentes \emph{hypervisors}, elaboração de ambientes de
alta disponibilidade de nós controladores, de processamento e de volumes.

Um importante trabalho futuro sequencial a este projeto está sendo elaborado no
momento. Servidores e equipamentos de rede de alta performance foram adquiridos
pela universidade através de um edital de apoio da FAPERJ para que a \emph{cloud}
seja disponibilizada para ser utilizada pelos próprios alunos e professores da
UENF. O uso inicial deste equipamento abrange oferecer hospedagem e ambiente
de testes para aplicações desenvolvidas pelos próprios alunos do curso de Ciência
da Computação, através de bolsas de iniciação científica e tecnológica.
