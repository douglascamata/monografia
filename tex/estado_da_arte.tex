\chapter{Trabalhos relacionados}

São vários os trabalhos que já apresentam casos de sucesso de uso de diversos
modelos e tecnologias de cloud computing. Alguns dão enfoque a instituições de
ensino situadas em países em desenvolvimento, como o trabalho de
\citeonline{book-cloud-developing-countries}. Ele descreve a implantação
de cloud computing na Universidade de Tecnologia de Ho Chi Minh, no Vietnã,
para as tarefas de pesquisa e ensino.

No trabalho de \citeonline{book-cloud-developing-countries},
recursos fixos e pré-configurados deixaram de ser fornecidos,
em prol de recursos elásticos (que podem ser facilmente redimensionados)
e móveis (podem estar inclusive em lugares físicos diferentes de acordo com a
vontade do seu utilizador). Os pesquisadores forneciam os dados
relacionados aos recursos que precisavam, seja de hardware ou software, e em
segundos obtinham o devido acesso que necessitavam (IaaS). Eles também podiam
ter acesso simples e direto a plataformas para desenvolverem suas pesquisas da
maneira mais prática possível (PaaS). Além disso, foi criada uma rede social
para que os pesquisadores compartilharem informações científicas mais facilmente
entre eles.

Em resumo, \citeonline{book-cloud-developing-countries} nos dizem que
o cloud computing solucionou um grande problema da
universidade, que eram as diferentes necessidades de sistemas operacionais e softwares,
muitas vezes conflitantes, e a massiva manutenção manual demandada para a
instalação e configurações dos softwares adequados a cada um dos cursos. A cada
mudança nas necessidades de software de um determinado curso, era necessário que
este fosse instalado e devidamente configurado em cada máquina do laboratório,
tomando muito tempo e esforço.
Com \emph{cloud computing}, as máquinas dos laboratórios passaram a ser servidas
através de \emph{IaaS}, onde cada curso possui uma imagem
relacionada a ele, que cumpre todos os requisitos de software, hardware virtual
e sistema operacional necessários. Quando há alguma modificação nesses
requisitos, ela é feita diretamente na imagem e esta é novamente disponibilizada
para todos os terminais de acesso. Facilita-se assim o gerenciamento do
laboratório, eliminando uma boa parte do excesso de trabalho manual e ainda
agilizando todo o processo de manutenção dos recursos computacionais físicos e
de software.

Anteriormente, em outro trabalho, \citeonline{article-cloud-education-new-dawn}
apresentou um estudo de caso da
Universidade de Westminster, no Reino Unido. Com mais de 22 mil alunos, ela é
uma de várias universidades desse país que aderiu ao cloud computing. Tudo
começou quando o serviço de email desta instituição começou a ser abandonado por
parecer desatualizado. Através de pesquisas, foi descoberto que 96\% dos
estudantes desta universidade configuravam suas contas de email acadêmico para
automaticamente redirecionar os emails para uma conta externa
\cite{article-cloud-education-new-dawn}. Isto estava causando alguns problemas,
dentre eles: os emails começaram a ser tratados como \textit{spam} pelos emails
externos dos estudantes e, consequentemente, eles estavam deixando de receber
comunicados importantes; e os alunos acabavam frequentemente usando pendrives,
que são muito suscetíveis a perdas ou mal-uso, porque o serviço não fornecia
uma capacidade de armazenamento adequada.

Segundo este mesmo autor, para solucionar o problema, em 2007 a universidade
começou a estudar o \textit{Google Apps: Education Edition}. Através dele,
seriam oferecidos email com 7.3gb de armazenamento (o suficiente para os alunos
abandonarem o uso de pendrives), um sistema de troca de mensagens, calendário
compartilhado e até uma suíte de escritório com aplicações para processamento de
textos, planilhas e apresentações. Durante o ano 2009/9, após um determinado
período de testes e \textit{feedback}, o sistema foi implantado e os
problemas apontados foram mitigados com sucesso. Os estudantes passaram a ter
uma melhor experiência e a universidade estima uma economia de 1 milhão
de libras esterlinas, caso fosse fornecer um serviço semelhante de forma
independente.

Ainda há outros dois casos de uso em instituições de ensino mencionados por
\citeonline{article-cloud-education-new-dawn}. O primeiro, da Universidade da
Califórnia, nos Estados Unidos da América, que está usando cloud computing
para seu curso de desenvolvimento e implantação de aplicações SaaS
(\textit{software as a service}). Ajudada por uma doação da \textit{Amazon Web
Services} (AWS), todo o curso foi movido para a nuvem da empresa e eles podem
fornecer a grande quantidade de servidores que o curso demanda em apenas alguns
minutos \cite{site-cloud-education}. O segundo é o caso do Colégio de Medicina
do Centro de Biotecnologia e Bioengenharia de Wisconsin, que está usando os
servidores de cloud computing do Google para tornar pesquisas relacionadas a
proteínas, que costumam ser caríssimas, disponíveis para cientistas em todo o
mundo.

Contudo, as aplicações de \emph{cloud computing} já estudadas não se limitam
ao setor empresarial, de forma geral, e de ensino. \citeonline{federal-cloud-computing}
nos mostra estudos sobre a sua utilização no Governo Federal dos Estados Unidos.
Foi planejado investir 25\% do orçamento do governo (cerca de 20 bilhões de dólares)
para migração de sistemas de todas as suas agências para \emph{cloud computing}.
Para auxiliar tais agências neste processo foi elaborada a Estratégia Federal
de Cloud Computing, com o objetivo de:

\begin{itemize}
    \item
        expor benefícios, considerações e dilemas do \emph{cloud computing};
    \item
        fornecer um framework de decisões e alguns casos de uso de exemplo
        para ajudar as agências a migrarem seus sistemas;
    \item
        destacar recursos para sua implementação;
    \item
        identificar atividades, paéis e responsabilidades do governo federal
        para catalisar a sua adoção.
\end{itemize}

Então elaborou-se um paralelo entre as principais características do ammbiente atual
e do ambiente de \emph{cloud computing} proposto, segundo 3 critérios: eficiência,
agilidade e inovação, que pode ser visto na Tabela \ref{table:federal-cloud-computing}.
Todas as agências, antes de migrar seus sistemas, deveriam fazer este comparativo
aplicado a cada sistema específico a avaliar se esta migração oferecerá vantagens consideráveis.
É incluso neste trabalho o exemplo do Centro de Experiências do Exército (AEC), onde foi identificada
a necessidade de atualizar seu sistema de gerencimanto de relação com o cliente e após
considerarem diversas opções, como atualizar seu sistema proprietário de 10 anos de idade,
escolherem uma solução de \emph{software} como serviço disponível comercialmente. Esta
solução foi adaptada para todos os requisitos de segurança desta agência, proporcionava
todas as funcionalides que eram necessárias e foi entregue a uma fração do tempo
e dinheiro necessários para atualizar o antigo sistema legado. Durante o processo de escolha
chegou-se às seguinte conclusões sobre os critérios propostos:

\begin{itemize}
    \item
    Eficiência: o custo inicial de atualizar o sistema legado era de 500 mil a 1 milhão
    de dólares, enquanto o investimento inicial da solução de \emph{software} como serviço
    teve o custo de 54 mil dólares;

    \item
    Agilidade: foi considerado também o tempo para atualização do antigo \emph{software}.
    Apesar dos upgrades ao longo dos anos, era inviável customizá-lo com as necessidades
    do centro. Enquanto isso, a nova solução oferece provisiosamento em uma fração do tempo
    do seu antecessor, além de ser mais escalável e fácil de atualizar conforme o tempo;

    \item
    Inovação: a solução de \emph{software} como serviço integra-se diretamente
    com email e \emph{Facebook} e acesso a informação de qualquer lugar, através
    da \emph{internet} e permite a agência aproveitar do sistema de inovações
    do seu provedor de serviços sem ter que gerenciar pesados recursos computacionais.

\end{itemize}

\begin{table}
    \caption{Benefícios do Cloud Computing (Fonte: \cite{federal-cloud-computing})}
    \label{table:federal-cloud-computing}
    \centering
    \begin{tabular}{|p{8cm}|p{8cm}|}
        \hline
        \multicolumn{2}{|c|}{Eficiência} \\
        \hline
        Cloud Computing & Ambiente atual \\
        \hline
        \tabitem Alta utilização de recursos (maior que 60-70\%) & \tabitem Baixa utilização de recursos (menor que 30\%) \\
        \tabitem Demanda agregada e sistema acelerado de consolidação & \tabitem Demanda fragmentada e sistemas duplicados \\
        \tabitem Produtividade no desenvolvimento e gerenciamento de aplicações, redes e usuários finais & \tabitem Difícil gerenciamento \\
        \hline
        \multicolumn{2}{|c|}{Agilidade} \\
        \hline
        Cloud Computing & Ambiente atual \\
        \hline
        \tabitem Comprado como serviço de fornecedores confiáveis & \tabitem Requer anos para construção de data-centers para novos serviços \\
        \tabitem Incrementos e decrementos de capacidade quase instantâneos & \tabitem Requer meses para aumentar a capacidade de serviços existentes \\
        \tabitem Mais responsivo a necessidades urgentes \\
        \hline
        \multicolumn{2}{|c|}{Inovação} \\
        \hline
        Cloud Computing & Ambiente atual \\
        \hline
        \tabitem Muda o foco da posse dos recursos para gerenciamento de serviços & \tabitem Sobrecarregado com gerenciamento de recursos \\
        \tabitem Incentiva inovação no setor privado & \tabitem Totalmente desacoplado de inovações no setor privado \\
        \tabitem Encoraja uma cultura empreendedora & \tabitem Sem riscos, uso de tecnologia legada \\
        \tabitem Mais ligado a tecnologias emergentes \\
        \hline
    \end{tabular}
\end{table}

De acordo com o trabalho de \citeonline{ibm-cloud-computing}, a IBM, grande
empresa do ramo de computação, conhecida desde os seus primórdios, tem um caso
de uso de sucesso absoluto de \emph{cloud computing}. Na empresa, esta tecnologia
é utilizada dentro das 6 principais cargas de trabalho de tecnologia da informação:
desenvolvimento e testes, análise, armazenamento, collaboração, computação pessoal
e aplicações em produção. Graças a este uso, segundo estes mesmos autores, a
IBM melhorou sua eficiência, enquanto ao mesmo obteve impressivas economias em
capital e operações. O trabaho apresenta as seguintes considerações:

\begin{itemize}
    \item
        As equipes de desenvolvimento e testes presenciaram o tempo de provisionamento
        de um servidor cair de 5 dias ou mais para apenas uma hora. Isto impactou
        num acelerado desenvolvimento de novas funcionalidades e na velocidade
        com que as aplicações desenvolvidas chegavam no mercado;

    \item
        A nuvem de análise de dados melhorou incrivelmente a área de \emph{business intelligence},
        e reduziu os custos de projetos desta área, que eram da magnitude de centenas de milhares de dólares.
        As organizações que utilizam esta nuvem tem acesso a informação agregada de
        centenas de \emph{warehouses} (deposítos). Estas informações tem altíssimo valor, chegando
        a 300 milhões de dólares, somente para as 20 maiores consumidoras, de um total de 300 organizações.

    \item
        Sua central de armazenamento baseada em \emph{cloud computing} cortou
        o custo de armazenamento por \emph{byte} em 50\%, permitindo a IBM
        acomodar o grande crescimento de demanda por armazenamento (estimado
        em 25\% ao ano) sem aumentar o orçamento total destinado a essa área.

    \item
        A rede social \emph{IBM Connections}, hospedada em sua própria \emph{cloud},
        aumentao dramasticamente a colaboração entre seus funcionários e, consequentemente,
        a produtividade e inovação destes. Esta rede social suporta cerca de 50 milhões de minutos
        em conferência por mês.
\end{itemize}

Antes da implantação de tecnologias de \emph{cloud computing}, vários gárgalos
e problemas eram encontrados na IBM. Dentre estes problemas, por exemplo,
o setor de desenvolvimento e testes enfrentava semanas de espera para obter um
servidor, utilizava somente 10\% dele e não liberava seu uso devido ao tempo de
espera para obtê-lo novamente. Neste trabalho é apontado que 95\% do provisionamento
dos servidores é agora feito através de \emph{cloud}, confirmando sua velocidade e
facilidade de uso.

\citeonline{impact-cloud-computing} faz uma análise sobre o assunto e chega a conclusão que
os diversos serviços baseados em \emph{cloud computing} já disponíveis fizeram com que as
pessoas mudassem a maneira como pensam sobre conteúdo digital e como utilizá-lo.
No setor empresarial, a implantação de \emph{cloud computing} está aumentando constantemente,
independente do nicho. O crescimento de serviços que
podem ser oferecidos neste modelo de \emph{utility computing} continuará a mudar o mercado,
apresentar novos modelos de negócios e revolucionar a maneira como compartilha-se informações.

A Doutora Tua Huomo, em \citeonline{impact-cloud-computing}, diz que \emph{cloud computing}
não implica em apenas uma mudança de tecnologia, mas também forçará todos a adotarem
processos operacionais e de desenvolvimento que são mais focados no valor para
o cliente. Neste mesmo trabalho, o Professor Ian Biterllin prediz que qualidades
únicas desta nova tecnologia tornam-a muito mais amigável ao meio ambiente
do que todos os outros paradigmas conhecidos, mitigando as grandes críticas
recebidas por \emph{data centers} quando ao seu consumo energético. Finalmente,
o Dr Jonathan Liebanau, da Escola de Economia de Londres, em sua análise chega
à conclusão que \emph{cloud computing} trará diversas mudanças positivas na economia, porém
cada país sentirá essas mudanças com uma intensidade diferente: elas dependerão de como
os provedores de serviço, o governo e os diretores de empresas se adaptarão.
