\begin{abstract}
This work intends to present a study of case of a cloud computing environment
in Universidade Estadual do Norte Fluminense (UENF), build using the open-source
tools known as OpenStack. It presents a bunch of common problems in public,
learning and research institutions related to their computational resources
management, already identified by many other authors, some concepts which
serve as base for cloud computing and how it can solve these problems. This
paper also presentes a little briefing of OpenStack's history, a report
about its instalation, the problems encountered, and the final interface
that is provided to the clients of this cloud.
\end{abstract}
