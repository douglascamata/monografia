\begin{resumo}
O objetivo deste trabalho é apresentar um estudo de caso de um ambiente de \emph{cloud computing},
realizado na Universidade Estadual do Norte Fluminense (UENF), construído através
do conjunto de ferramentas de código livre conhecido como \emph{OpenStack}. Esta
monografia apresenta um apanhado de problemas comuns em instituições públicas, de
ensino e pesquisa com relação a gerencia de seus recuros computacionais, já identificados por
diversos autores, alguns conceitos que servem de base para o \emph{cloud computing}
e como ele pode resolver estes problemas. Também é apresentado um resumo sobre a história
do \emph{OpenStack}, um relato sobre sua instalação, alguns problemas enfrentados
e a interface final que é fornecida para os utilizadores desta \emph{cloud}.
\end{resumo}
